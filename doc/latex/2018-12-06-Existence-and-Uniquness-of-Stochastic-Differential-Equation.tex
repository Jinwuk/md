\PassOptionsToPackage{unicode=true}{hyperref} % options for packages loaded elsewhere
\PassOptionsToPackage{hyphens}{url}
%
\documentclass[]{article}
\usepackage{lmodern}
\usepackage{amssymb,amsmath}
\usepackage{ifxetex,ifluatex}
\usepackage{fixltx2e} % provides \textsubscript
\ifnum 0\ifxetex 1\fi\ifluatex 1\fi=0 % if pdftex
  \usepackage[T1]{fontenc}
  \usepackage[utf8]{inputenc}
  \usepackage{textcomp} % provides euro and other symbols
\else % if luatex or xelatex
  \usepackage{unicode-math}
  \defaultfontfeatures{Ligatures=TeX,Scale=MatchLowercase}
\fi
% use upquote if available, for straight quotes in verbatim environments
\IfFileExists{upquote.sty}{\usepackage{upquote}}{}
% use microtype if available
\IfFileExists{microtype.sty}{%
\usepackage[]{microtype}
\UseMicrotypeSet[protrusion]{basicmath} % disable protrusion for tt fonts
}{}
\IfFileExists{parskip.sty}{%
\usepackage{parskip}
}{% else
\setlength{\parindent}{0pt}
\setlength{\parskip}{6pt plus 2pt minus 1pt}
}
\usepackage{hyperref}
\hypersetup{
            pdftitle={"Existence and Uniqueness of Stochastic Differential Equation"},
            pdfborder={0 0 0},
            breaklinks=true}
\urlstyle{same}  % don't use monospace font for urls
\usepackage{color}
\usepackage{fancyvrb}
\newcommand{\VerbBar}{|}
\newcommand{\VERB}{\Verb[commandchars=\\\{\}]}
\DefineVerbatimEnvironment{Highlighting}{Verbatim}{commandchars=\\\{\}}
% Add ',fontsize=\small' for more characters per line
\newenvironment{Shaded}{}{}
\newcommand{\AlertTok}[1]{\textcolor[rgb]{1.00,0.00,0.00}{\textbf{#1}}}
\newcommand{\AnnotationTok}[1]{\textcolor[rgb]{0.38,0.63,0.69}{\textbf{\textit{#1}}}}
\newcommand{\AttributeTok}[1]{\textcolor[rgb]{0.49,0.56,0.16}{#1}}
\newcommand{\BaseNTok}[1]{\textcolor[rgb]{0.25,0.63,0.44}{#1}}
\newcommand{\BuiltInTok}[1]{#1}
\newcommand{\CharTok}[1]{\textcolor[rgb]{0.25,0.44,0.63}{#1}}
\newcommand{\CommentTok}[1]{\textcolor[rgb]{0.38,0.63,0.69}{\textit{#1}}}
\newcommand{\CommentVarTok}[1]{\textcolor[rgb]{0.38,0.63,0.69}{\textbf{\textit{#1}}}}
\newcommand{\ConstantTok}[1]{\textcolor[rgb]{0.53,0.00,0.00}{#1}}
\newcommand{\ControlFlowTok}[1]{\textcolor[rgb]{0.00,0.44,0.13}{\textbf{#1}}}
\newcommand{\DataTypeTok}[1]{\textcolor[rgb]{0.56,0.13,0.00}{#1}}
\newcommand{\DecValTok}[1]{\textcolor[rgb]{0.25,0.63,0.44}{#1}}
\newcommand{\DocumentationTok}[1]{\textcolor[rgb]{0.73,0.13,0.13}{\textit{#1}}}
\newcommand{\ErrorTok}[1]{\textcolor[rgb]{1.00,0.00,0.00}{\textbf{#1}}}
\newcommand{\ExtensionTok}[1]{#1}
\newcommand{\FloatTok}[1]{\textcolor[rgb]{0.25,0.63,0.44}{#1}}
\newcommand{\FunctionTok}[1]{\textcolor[rgb]{0.02,0.16,0.49}{#1}}
\newcommand{\ImportTok}[1]{#1}
\newcommand{\InformationTok}[1]{\textcolor[rgb]{0.38,0.63,0.69}{\textbf{\textit{#1}}}}
\newcommand{\KeywordTok}[1]{\textcolor[rgb]{0.00,0.44,0.13}{\textbf{#1}}}
\newcommand{\NormalTok}[1]{#1}
\newcommand{\OperatorTok}[1]{\textcolor[rgb]{0.40,0.40,0.40}{#1}}
\newcommand{\OtherTok}[1]{\textcolor[rgb]{0.00,0.44,0.13}{#1}}
\newcommand{\PreprocessorTok}[1]{\textcolor[rgb]{0.74,0.48,0.00}{#1}}
\newcommand{\RegionMarkerTok}[1]{#1}
\newcommand{\SpecialCharTok}[1]{\textcolor[rgb]{0.25,0.44,0.63}{#1}}
\newcommand{\SpecialStringTok}[1]{\textcolor[rgb]{0.73,0.40,0.53}{#1}}
\newcommand{\StringTok}[1]{\textcolor[rgb]{0.25,0.44,0.63}{#1}}
\newcommand{\VariableTok}[1]{\textcolor[rgb]{0.10,0.09,0.49}{#1}}
\newcommand{\VerbatimStringTok}[1]{\textcolor[rgb]{0.25,0.44,0.63}{#1}}
\newcommand{\WarningTok}[1]{\textcolor[rgb]{0.38,0.63,0.69}{\textbf{\textit{#1}}}}
\usepackage{graphicx,grffile}
\makeatletter
\def\maxwidth{\ifdim\Gin@nat@width>\linewidth\linewidth\else\Gin@nat@width\fi}
\def\maxheight{\ifdim\Gin@nat@height>\textheight\textheight\else\Gin@nat@height\fi}
\makeatother
% Scale images if necessary, so that they will not overflow the page
% margins by default, and it is still possible to overwrite the defaults
% using explicit options in \includegraphics[width, height, ...]{}
\setkeys{Gin}{width=\maxwidth,height=\maxheight,keepaspectratio}
\setlength{\emergencystretch}{3em}  % prevent overfull lines
\providecommand{\tightlist}{%
  \setlength{\itemsep}{0pt}\setlength{\parskip}{0pt}}
\setcounter{secnumdepth}{0}
% Redefines (sub)paragraphs to behave more like sections
\ifx\paragraph\undefined\else
\let\oldparagraph\paragraph
\renewcommand{\paragraph}[1]{\oldparagraph{#1}\mbox{}}
\fi
\ifx\subparagraph\undefined\else
\let\oldsubparagraph\subparagraph
\renewcommand{\subparagraph}[1]{\oldsubparagraph{#1}\mbox{}}
\fi

% set default figure placement to htbp
\makeatletter
\def\fps@figure{htbp}
\makeatother


\title{"Existence and Uniqueness of Stochastic Differential Equation"}
\date{2019-01-27 01:45:00 +0900}

\begin{document}
\maketitle

\tableofcontents

This article illustrates the existence and uniqueness of stochastic
differential equations. In the procedure of the proof, there are many
techniques and ideas for the analysis of the dynamics described by a
stochastic differential form. Most of contents in this articles are
strongly depending on the reference{[}1{]}.

\hypertarget{header-n4}{%
\subsection{Strong Solution}\label{header-n4}}

Let the stochastic differential equation as follows:

\[dX_t = a(t, X_t) dt + b(t, X_t)dW_t
\label{eq01:strong}
\tag{1}\]

The integral form of \(\eqref{eq01:strong}\) is

\[X_t = X_{t_0} + \int_{t_0}^t a(s, X_s)ds + \int_{t_0}^t b(S, X_s) dW_s
\label{eq02:strong}
\tag{2}\]

\hypertarget{header-n9}{%
\subsubsection{Lemma : Uniqueness}\label{header-n9}}

If assumptions of the measurable and Lipschitz continuous in
\href{https://jinwuk.github.io/mathematics/stochastic\%20calculus/2018/11/25/Stochastic-Calculus-Important_Assumption_and_Lemma.html\#borel-cantelli-lemma}{Important
Assumption and Lemma for Stochastic Analysis} are hold, then the
solution of \(\eqref{eq02:strong}\) corresponding to \textbf{the same
initial value and the same Wiener process are path-wise unique.}

\hypertarget{header-n11}{%
\paragraph{proof of Lemma}\label{header-n11}}

For \(N>0\) and \(t \in [t_0, T]\), we define the indication function
such that

\[I_t^{(N)}(w) = 
\begin{cases}
1 &: |X_u(w)|, |\tilde{X}_u(w)| \leq N \;\; \text{for} t_0 \leq u \leq t \\
0 &: \text{otherwise}
\end{cases}
\tag{3}\]

Obviously \(I_t^{(N)}\) is \(\mathcal{A}_t\)-measurable and
\(I_t^{(N)} = I_t^{(N)} I_s^{(N)}\) for \(t_0 \leq u \leq t\).

Let \(Z_t^{(N)} = I_t^{(N)} (X_t - \tilde{X}_t)\)

\[Z_t^{(N)} 
= I_t^{(N)} \int_{t_0}^t I_s^{(N)} \left( a(s, X_s) - a (s, \tilde{X}_s) ds \right) 
+ I_t^{(N)} \int_{t_0}^t I_s^{(N)} \left( b(s, X_s) - b(s, \tilde{X}_s) dW_s \right)
\tag{4}\]

By the Lipschitz condition, for \(t_0 \leq s \leq t\),

\[\max \left\{ \left| a(s, X_s) - a(s, \tilde{X}_s)\right|, \left| b(s, X_s) - b(s, \tilde{X}_s) \right| \right\}

\leq K I_s^{(N)} \left| X_s - \tilde{X}_s \right| \leq 2KN
\label{eq01:pf_lemma}
\tag{5}\]

Thus the following inequality is evaluated (and see the \textbf{Note 1})

\begin{aligned}
\mathbb{E} \left( \left| Z_t^{(N)} \right|^2 \right)

&\leq  2 \mathbb{E} \left( \left| \int_{t_0}^t I_s^{(N)} \left( a(s, X_s) - a(s, \tilde{X}_s) \right) ds \right|^2 \right) \\

&+ 2 \mathbb{E} \left( \left| \int_{t_0}^t I_s^{(N)} \left( b(s, X_s) - b(s, \tilde{X}_s) \right) dW_s \right|^2 \right) \\

&\leq  2 (T - t_0) \int_{t_0}^t \mathbb{E} \left( \left| I_s^{(N)} \left( a(s, X_s) - a(s, \tilde{X}_s) \right) \right|^2 \right) ds \\

&+ 2\int_{t_0}^t \mathbb{E} \left( \left| \left( b(s, X_s) - b(s, \tilde{X}_s) \right) \right|^2 \right) ds \;\;\; \because \text{Note 1} \\

&\leq  2 (T - t_0) K^2 \int_{t_0}^t \mathbb{E} \left( I_s^{(N)} \left| X_s - \tilde{X}_s \right|^2 \right) ds \\

&+ 2 K^2 \int_{t_0}^t \mathbb{E} \left( I_s^{(N)} \left| X_s - \tilde{X}_s \right|^2 \right) ds \;\;\; \because \text{By Lipschitz conditiion } \eqref{eq01:pf_lemma} \\

&\leq  2 (T - t_0 + 1) K^2 \int_{t_0}^t \mathbb{E} \left( I_s^{(N)} \left| X_s - \tilde{X}_s \right|^2 \right) ds 

\end{aligned}

Therefore, we can obtain

\[\mathbb{E} \left( \left| Z_t^{(N)} \right|^2 \right) \leq L \int_{t_0}^t \mathbb{E} \left( \left| Z_s^{(N)}  \right|^2 \right)
\label{eq02:pf_lemma}
\tag{6}\]

for \( t \in [t_0, T]\) where \(L = 2(T - t_0 + 1)K^2 \).

Apply \textbf{Grownwall's Lemma} with

\[\alpha(t) = \mathbb{E}\left( \left| Z_t^{(N)} \right|^2 \right)\]

and \(\beta(t) \equiv 0\), then

\[\alpha(t) \leq \beta(t) + L \int_{t_0}^t e^{L(t-s)}\beta(s) ds
= 0.
\tag{7}\]

We conclude that

\[\mathbb{E} \left( \left| Z_t^{(N)} \right|^2 \right) 
= \mathbb{E} \left( \left| I_t^{(N)} (X_t - \tilde{X}_t) \right|^2 \right) = 0.
\tag{8}\]

In addition, for \(t \in [t_0, T]\),
\(I_t^{(N)} X_t = I_t^{(N)} \tilde{X}_t\) with probability 1.

Since \textbf{the sample paths are continuous almost surely}, they are
bounded almost surely. Thus we can make the probability

\[P \left( I_t^{(N)} \not\equiv 1, \forall t \in [t_0, T] \right) \leq P \left( \sup_{t_0 \leq t \leq T} |X_t| > N \right) + P \left( \sup_{t_0 \leq t \leq T} |\tilde{X}_t| > N \right)
\tag{9}\]

arbitrarily small by taking \(N\) sufficiently large. This means that
\(P \left( X_t \neq \tilde{X}_t \right) = 0\) for each
\(t \in [t_0, T] \), and hence that
\(P \left( X_t \neq \tilde{X}_t : t \in D \right) = 0\) for any
countable dense subset \(D\) of \([t_0, T]\).

\textbf{Q.E.D}

\hypertarget{header-n35}{%
\paragraph{Meaning of Superscript (N)}\label{header-n35}}

See the following diagram.
\includegraphics{https://drive.google.com/uc?id=1k48IbTc1JLy-Gt4-7vPDDPonmk9y_oKq}

즉, 위 그림에서 처럼, 어떤 단계를 Stochastic process가 도달 했는지
아닌지를 알리는 것이다.

이는 Lebesgue Integral을 위해 필요한 단계이다.

\hypertarget{header-n41}{%
\subsubsection{Theorem : Strong Solution}\label{header-n41}}

Under assumptions in the article,
\href{https://jinwuk.github.io/mathematics/stochastic\%20calculus/2018/11/25/Stochastic-Calculus-Important_Assumption_and_Lemma.html\#borel-cantelli-lemma}{Important
Assumption and Lemma for Stochastic Analysis}, the stochastic
differential equation \(\eqref{eq01:strong}\) has a path-wise unique
strong solution \(X_t\) on \([t_0, T]\) with

\[\sup_{t_0 \leq t \leq T} \mathbb{E} \left( |X_t |^2 \right) < \infty
\tag{10}\]

\hypertarget{header-n44}{%
\paragraph{proof}\label{header-n44}}

Let \(X_t^{(0)} \equiv X_{t_0}\) and

\[X_t^{(n+1)} = X_{t_0} + \int_{t_0}^t a(s, X_s^{(n)})ds + \int_{t_0}^t b(S, X_s^{(n)}) dW_s
\label{eq01:strong_pf}
\tag{11}\]

for \(n = 0, 1, 2, \cdots \).

For assumption of the initial value, it is clear that

\[\sup_{t_0 \leq t \leq T} E\left( |X_t^{(0)} |^2 \right) < \infty
\tag{12}\]

Applying the inequality \((a + b +c)^2 \leq 3(a^2 + b^2 + c^2)\), the
Cauchy-Schwarz inequality and the linear growth bound to
\(\eqref{eq01:strong_pf}\),

\textbf{Note 1} By the Cauchy-Schwartz inequality,

\begin{aligned}
\left| \int_{t_0}^T A(s) ds \right|^2 
&\leq \int_{t_0}^T \left| A(s) \right| ds \cdot \int_{t_0}^T \left| A(s) \right| ds \\
&\leq \int_{t_0}^T \left| A(s) \right|^2 ds \cdot \int_{t_0}^T  ds 
\;\;\; \because |a|^2 \leq |A|^3
\end{aligned}

Therefore,

\[\left| \int_{t_0}^T A(s) ds \right|^2 \leq (T - t_0) \int_{t_0}^T \left| A(s) \right|^2 ds
\tag{13}\]

\textbf{End of Note 1}

we obtain

\begin{aligned}
\mathbb{E} \left( \left| X_t^{(n+1)} \right|^2 \right)
&\leq 3 \mathbb{E} \left( \left| X_{t_0} \right|^2 \right) + 3 \mathbb{E} \left( \left| \int_{t_0}^t a(s, X_s^{(n)}) ds \right|^2 \right) + 3 \mathbb{E} \left( \left| \int_{t_0}^t b(S, X_s^{(n)}) dW_s \right|^2 \right)  \\
&\leq 3 \mathbb{E} \left( \left| X_{t_0} \right|^2 \right) + 3 (T - t_0) \left( \mathbb{E} \left(  \int_{t_0}^t \left| a(s, X_s^{(n)}) \right|^2 ds  \right) +  \mathbb{E} \left( \int_{t_0}^t \left| b(S, X_s^{(n)}) \right|^2 ds  \right) \right)\\
&\leq 3 \mathbb{E} \left( \left| X_{t_0} \right|^2 \right) + 3 (T - t_0 + 1) K^2 \mathbb{E} \left( \int_{t_0}^t \left( 1 + \left| X_s^{(n)} \right|^2 \right) ds  \right)
\end{aligned}

for \(n=0, 1, 2, \cdots \). By induction we thus have

\[\sup_{t_0 \leq t \leq T} \mathbb{E} \left( \left| X_t^{(n)} \right|^2\right) \leq C_0 < \infty
\tag{14}\]

for \(n=0, 1, 2, \cdots \).

이로써
\textbf{\(\mathbb{E} \left( {\mid X_t^{(n+1)} \mid }^2 \right)\)의
Finite 특성이 증명}되었음. 이를 토대로 \textbf{Lemma를 적용}한다.

\textbf{Note 2} 앞에서
\(\mathbb{E} \left( {\mid X_t^{(n+1)} \mid }^2 \right)\) 가 증명
되었으므로 Lemma 에서의 식 \(\eqref{eq02:pf_lemma}\) 에서 다음의 특성을
얻는다.

\[\mathbb{E} \left( \left| X_t^{(n+1)} - X_t^{(n)} \right|^2 \right)
\leq L \int_{t_0}^t \mathbb{E} \left( \left| X_s^{(n+1)} - X_s^{(n)} \right|^2 \right) ds\]

for \( t \in [t_0, T]\) and \(n=1, 2, 3 \cdots \) where
\(L = 2(T - t_0 + 1)K^2\).

Cauchy Formula \(\eqref{eq01:appd}\)에 의해
\(\mathbb{E} \left( \left| X_t^{(n+1)} - X_t^{(n)} \right|^2 \right)\)
와
\(\mathbb{E} \left( \left| X_s^{(n+1)} - X_s^{(n)} \right|^2 \right)\)
간의 관계식은 다음과 같다. (즉 Level이 \(X_s^{(0)}\) 에서 \(X_s^{(n)}\)
까지 움직여야 한다. Cauchty Formula의 자세한 내용은 Appendix를
참조한다.)

By Cauchy Formula, We obtain

\[\mathbb{E} \left( \left| X_t^{(n+1)} - X_t^{(n)} \right|^2 \right)
\leq \frac{L^n}{(n-1)!} \int_{t_0}^t (t-s)^{n-1} \mathbb{E} \left( \left| X_s^{(1)} - X_s^{(0)} \right|^2 \right) ds
\label{eq02:strong_pf}
\tag{14}\]

for \(t \in [t_0, T], \; n=0, 1, 2, \cdots \).

Herein, we use \textbf{Growth bound} Condition, instead of Lipschitz
Condition, (look at the
\href{https://jinwuk.github.io/mathematics/stochastic\%20calculus/2018/11/25/Stochastic-Calculus-Important_Assumption_and_Lemma.html}{Link
of Assumption} ) in the derivation of \(\eqref{eq02:strong_pf}\), then,
for \(n=0\), we find that

\begin{aligned}
\mathbb{E} \left( \left| X_t^{(n+1)} - X_t^{(n)} \right|^2 \right) 

&\leq L \int_{t_0}^t \left( 1 + \mathbb{E} \left( \left | X_s^{(0)} \right |^2   \right) \right) \\

&\leq L (T - t_0) \left( 1 + \mathbb{E} \left( \left | X_s^{(0)} \right |^2   \right) \right) = C_1

\end{aligned}

On inserting this into \(\eqref{eq02:strong_pf}\), we obtain

\[\mathbb{E} \left( \left| X_t^{(n+1)} - X_t^{(n)} \right|^2 \right) 
\leq \frac{C_1 L^n (t - t_0)^n}{n!}\]

for \(t \in [t_0, T], \; n = 0, 1, 2,  \cdots\)m and hence

\[\sup_{t_0 \leq t \leq T} \mathbb{E} \left( \left| X_t^{(n+1)} - X_t^{(n)} \right|^2 \right) 
\leq \frac{C_1 L^n (t - t_0)^n}{n!}
\label{eq03:strong_pf}
\tag{15}\]

이 자체로 \textbf{the mean-suqare convergence of the successive
approximation uniformly} 를 의미한다.

하지만 우리가 원하는 것은 \textbf{the almost sure convergence of their
sample paths uniformly} 이다.

이를 위해서는 최종적으로 이것이, 최소한 \textbf{Chebyshev 부등식을
만족}함을 보여야 한다. 그러기 위해서는 다음의 \textbf{Square 값의
Supremum을 계산}하여야 한다.

To show this, we define

\[Z_n = \sup_{t_0 \leq t \leq T} \left | X_t^{n+1} - X_t^{n}  \right |\]

For \(n=\mathbf{Z}[0, \infty]\) From \(\eqref{eq01:pf_lemma}\), we
obtain

\[Z_n 
\leq \int_{t_0}^t \left | a(s, X_s^{n}) - a(s, X_s^{n-1}) \right | ds 
+ \sup_{t_0 \leq t \leq T} \left | \int_{t_0}^t \left( b(s, X_s^{(n)} - b(s, X_s^{(n-1)}) \right) \right | dW_s.\]

By Lipschitz Condition

\begin{aligned}
\left | a(s, X_s^{(n)}) - a(s, X_s^{(n-1)}) \right | &\leq K \left |X_s^{(n)} -X_s^{(n-1)} \right | \\

\left | b(s, X_s^{(n)}) - b(s, X_s^{(n-1)}) \right | &\leq K \left | X_s^{(n)} -X_s^{(n-1)} \right | 

\end{aligned}

By Cauchy-Schawartz

\begin{aligned}
\left | a + b \right |^2 &\leq 2 \left | a \right |^2 + 2 \left | b \right |^2 \\
\left(\int_{t_0}^T a(s) ds \right)^2 
&= \left(\int_{t_0}^T a^2(s) ds \right)\left(\int_{t_0}^T a^2(s) ds \right) \\
&\leq \int_{t_0}^T a^2(s) ds \int_{t_0}^T ds = (T-t_0) \int_{t_0}^T a^2(s) ds
\end{aligned}

By Doob's Inequality

\[\mathbb{E} \sup_{0 \leq s \leq t} \left | b(s, X_s^{(n)}) - b(s, X_s^{(n-1)}) \right |^2 \leq \left( \frac{2}{2-1} \right)^2 \mathbb{E} \left | b(s, X_s^{(n)}) - b(s, X_s^{(n-1)}) \right |^2\]

Subsequently,

\begin{aligned}
\mathbb{| Z_n |^2} &\leq 2(T - t_0) K^2 \int_{t_0}^T \mathbb{E} \left( \left | X_s^{(n)} -X_s^{(n-1)}  \right | \right) ds
+ 8K^2 \int_{t_0}^T \mathbb{E} \left( \left | X_s^{(n)} -X_s^{(n-1)}  \right | \right) ds \\
& \leq 2(T - t_0 + 4)K^2 \int_{t_0}^T \mathbb{E} \left( \left | X_s^{(n)} -X_s^{(n-1)}  \right | \right) ds 
\end{aligned}

We combined with \(\eqref{eq03:strong_pf}\) to conclude that

\[\mathbb{E}| Z_n |^2 \leq \frac{C_1 L^n (t - t_0)^n}{n!}\]

where \(C_2 = C_1 K^2 (T - t_0 + 4)(T - t_0)\) for
\(n = 1, 2, 3,  \cdots\).

After applying the Markov inequality, i.e.

\[P(Z_n > \frac{1}{n^2}) \leq \left( n^2 \right) \mathbb{E}|Z_n|^2\]

to each term and summing, we have

\[\sum_{n=1}^{\infty} P(Z_n > \frac{1}{n^2}) \leq C_2 \sum_{n=1}^{\infty} \frac{n^4}{(n-1)!} L^{n-1} (T - t_0)^{n-1}\]

Checking the internal term of summation in the right-side , we evaluate
the ratio of order such that

\[\frac{e_n}{e_{n-1}} = \frac{\frac{n^4}{(n-1)!} L^{n-1} (T - t_0)^{n-1}}{ \frac{ {(n-1)}^4 }{(n-2)!} L^{n-2} (T - t_0)^{n-2}} = \frac{n^4}{(n-1)^5} L (T - t_0).\]

When \( n > 4 \) , \(\frac{e_n}{e_{n-1}}  < 1\) and \(n^4 < (n -1)!\).
Therefore the summation can be converges for \(n > n_0\) where
\(n_0 > 0\) is a sufficient large.

Hence the series on the left side also converges, so by the
\textbf{Borel-Cantelli Lemma}, \(Z_n\) converges 0, almost surely. that
is \textbf{the successive approximations \(X_s^{(n)} \) converges almost
surely}, uniformly on \([t_0, T]\) to a limit \(\tilde{X}_t\) defined by

\[\tilde{X}_t = X_{t_0} + \sum_{n=0}^{\infty} \left( X_t^{(n+1)} - X_t^{(n)}\right)\]

즉, 여기서 \(\sum_{n=0}^{\infty} \left( X_t^{(n+1)} - X_t^{(n)}\right)\)
가 확률적으로 유한하므로 \(\left( X_t^{(n+1)} - X_t^{(n)}\right)\) 는
\(n \uparrow \infty\) 에 따라 0에 수렴해야 한다.

As the limit of \(\mathcal{A}^*\) -adapted processes, \(\tilde{X}\) is
\(\mathcal{A}^*\) -adapted and as the uniform limit of continuous, it is
continuous

따라서, \(n \rightarrow \infty\) 에 따라 Lipschitz Continuous 에 의해

\begin{aligned}
\left | \int_{t_0}^t a(s, X_s^{(n)}) ds - \int_{t_0}^t a(s, \tilde{X}_s^{(n)}) ds \right | 
&\leq K \int_{t_0}^t \left | X_s^{(n)} - \tilde{X}_s^{(n)} \right | ds \rightarrow 0 \\
\int_{t_0}^t \left | b(s, X_s^{(n)}) - b(s, \tilde{X}_s)  \right |^2 ds 
&\leq K^2 \int_{t_0}^t \left | X_s^{(n)} - \tilde{X}_s \right |^2 ds \rightarrow 0
\end{aligned}

with probability 1, which implies

\begin{aligned}
\int_{t_0}^t a(s, X_s^{(n)}) ds &\rightarrow \int_{t_0}^t a(s, \tilde{X}_s^{(n)}) ds \\
\int_{t_0}^t b(s, X_s^{(n)}) dW_s &\rightarrow \int_{t_0}^t b(s, \tilde{X}_s)   dW_s
\end{aligned}

in probability, as \(n \rightarrow \infty\) for each \(t \in [t_0, T] \)
. Hence the right side of \(\eqref{eq01:strong_pf}\) converges to the
right side of \(\eqref{eq02:strong}\), and so the limit process
\(\tilde{X}\) satisfies the stochastic integral equation
\(\eqref{eq02:strong}\).

This completes the proof of the existence and uniqueness of a strong
solution of the stochastic differential equation \(\eqref{eq01:strong}\)
for an initial value \(X_{t_0}\) with
\(\mathbb{E}( \mid X_{t_0}  \mid^2) < \infty\).

\hypertarget{header-n110}{%
\subsubsection{Conclusion}\label{header-n110}}

\begin{itemize}
\item
  다시말해, Assumption A1\textasciitilde{}A4 를 만족하면,
  (Measurabilitry, Lipschitz Continuous, Linear Growth, Finite Initial
  Value) 만 만족하면, \textbf{pathwise unique strong solution \(X_t\)}
  가 \([t_0, T]\) 에서 존재한다는 것이다.
\item
  \(\eqref{eq03:strong_pf}\) 조건이 최근 여러 논문에서 자주 언급된다. 
\end{itemize}

\[\sup_{t_0 \leq t \leq T} \mathbb{E} \left( \left| X_t^{(n+1)} - X_t^{(n)} \right|^2 \right) 
\leq \frac{C_1 L^n (t - t_0)^n}{n!}\]

\begin{itemize}
\item
  생각해 보면 이 조건은 다음 Stochastic integral 에서
  \(\eqref{eq01:strong_pf}\) 의 solution \(X_t^{(n+1)}\) 이 존재한다는
  것을 증명하는 것으로 \(n \uparrow \infty\) 에서 사실상
  \(X_t^{(n)} \rightarrow  X_t^{(n+1)}\) with \textbf{the mean-square
  convergence of the successive approximation uniformly} 특성을 통해
  다음이 \textbf{Unique 하게 존재한다는 것을 Strong Sense로 증명}한
  것이다.

  \begin{aligned}
  X_t^{(n+1)} 
  &= X_{t_0} + \int_{t_0}^t a(s, X_s^{(n)})ds + \int_{t_0}^t b(S, X_s^{(n)}) dW_s \\
  \Rightarrow
  X_t &= X_{t_0} + \int_{t_0}^t a(s, X_s)ds + \int_{t_0}^t b(S, X_s) dW_s
  \end{aligned}
\item
  그러므로 알고리즘의 수렴성을 증명하기 위해서는 이러한 Bounded 조건이
  만족됨을 보이고 (최소한 Reference를 통해서 ... ) 그 다음에 본격적으로
  수렴성을 증명해야 한다.
\item
  Therefore, to prove the convergence of any machine learning algorithm,
  researchers have to verify such a bounded condition of
  \(\mathbb{E}(\mid X_t \mid )^2 < \infty, \;\; \forall t \in [t_0, T]\)
  ( at least, by reference). Following this, they have to prove the main
  convergence property of any proposed algorithm, additionally.
\end{itemize}

\hypertarget{header-n126}{%
\subsection{Appendix}\label{header-n126}}

\hypertarget{header-n127}{%
\subsubsection{Cauchy Formula}\label{header-n127}}

\[\int_{t_0}^t \int_{t_0}^{t_{n-1}} \cdots \int_{t_0}^{t_1} f(s) ds dt_1 \cdots dt_{n-1} = \frac{1}{n-1} \int_{t_0}^t (t-s)N{n-1} f(s) ds
\label{eq01:appd}
\tag{A1}\]

\hypertarget{header-n129}{%
\subsubsection{Maximal Martingale inequality}\label{header-n129}}

A seperable martingale \(X = \{ X_t, t \geq 0 \}\) with finite p-th
moment satisfies the maximal martingale inequality

\[P \left( \left\{ w \in \Omega : \sup_{0 \leq s \leq t} \left | X_s (w) \right | \geq a \right\} \right) \leq \frac{1}{a^p} \mathbb{E} \left( \left | X_t \right |^p \right)
\label{eq02:appd}
\tag{A2}\]

\hypertarget{header-n132}{%
\subsubsection{Doob's Inequality}\label{header-n132}}

\[\mathbb{E} \left( \sup_{0 \leq s \leq t}  \left | X_t \right |^p \right)
\leq \left( \frac{p}{p-1} \right)^p \mathbb{E} \left( \left | X_t \right |^p \right)
\label{}
\label{eq03:appd}
\tag{A3}\]

\hypertarget{header-n134}{%
\subsubsection{Borel Cantelli Lemma}\label{header-n134}}

For any sequence of events
\(A_1, A_2, \cdots A_n, \cdots \in \mathcal{A}\)

\[P \left( \bigcap_{n=1}^{\infty} \bigcup_{k=n}^{\infty} A_k \right) = 0 \;\;\text{if } \sum_{n=1}^{\infty} P(A_n) < \infty\]

\hypertarget{header-n137}{%
\subsubsection{Corollary }\label{header-n137}}

Suppose that A1-A4 (Fundamental Assumptions in
\href{https://jinwuk.github.io/mathematics/stochastic\%20calculus/2018/11/25/Stochastic-Calculus-Important_Assumption_and_Lemma.html}{the
link} ) hold and that

\[\mathbb{E} \left( | X_{t_0} |^{2n} \right) < \infty\]

for some integer \( n \geq 1\). Then the solution \(X_t\) of
\(\eqref{eq01:strong}\) satisfies

\[\mathbb{E}\left(  \left | X_{t_0}  \right |^{2n} \right) \leq \left( 1 + \mathbb{E} \left( \left | X_{t_0} \right |^{2n} \right)\right) e^{C(t - t_0)}
\label{eq01:corollary}
\tag{C1}\]

and

\[\mathbb{E}\left(  \left |X_t -  X_{t_0}  \right |^{2n} \right) \leq D \left( 1 + \mathbb{E} \left( \left | X_{t_0} \right |^{2n} \right)\right) (t -t_0 )^n e^{C(t - t_0)}
\label{eq02:corollary}
\tag{C2}\]

for \(t \in [t_0, T]\) where \(T < \infty, \; C = 2n (2n + 1) K^2\) and
\(D\) is a positive constant depending only on \(K\) and \(T - t_0\).

\hypertarget{header-n145}{%
\paragraph{Note}\label{header-n145}}

The above Collorlary illustrates that the bounded condition
\(\eqref{eq03:strong_pf}\) is not suffcient for the convergence. Suppose
the case of \(n=1\), the bound of
\(\mathbb{E}\left(  \left |X_t -  X_{t_0}  \right |^{2n} \right)\) is
diverge as \(t\) is monotone increased to \(\infty\). There is not any
control parameter the increasing time parameter in
\(\eqref{eq02:corollary}\).

\begin{itemize}
\item
  The proof is very similar to that of theorem 1.

  \begin{itemize}
  \item
    Therefore, it requires an additional analysis of the algorithm.
  \end{itemize}
\item
  Using H\textbackslash{}older inequality and Grownwell's lemma
  additionally. 
\end{itemize}

\hypertarget{header-n156}{%
\subsection{References}\label{header-n156}}

{[}1{]}

\begin{Shaded}
\begin{Highlighting}[]
\VariableTok{@book}\NormalTok{\{}\OtherTok{kloeden2011numerical}\NormalTok{,}
  \DataTypeTok{title}\NormalTok{=\{Numerical Solution of Stochastic Differential Equations\},}
  \DataTypeTok{author}\NormalTok{=\{Kloeden, P.E. and Platen, E.\},}
  \DataTypeTok{isbn}\NormalTok{=\{9783540540625\},}
  \DataTypeTok{lccn}\NormalTok{=\{92015916\},}
  \DataTypeTok{series}\NormalTok{=\{Stochastic Modelling and Applied Probability\},}
  \DataTypeTok{url}\NormalTok{=\{https://books.google.co.kr/books?id=BCvtssom1CMC\},}
  \DataTypeTok{year}\NormalTok{=\{2011\},}
  \DataTypeTok{publisher}\NormalTok{=\{Springer Berlin Heidelberg\}}
\NormalTok{\}}
\end{Highlighting}
\end{Shaded}

\end{document}
